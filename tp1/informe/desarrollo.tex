Como ya adelantamos en la seccion anterior el objetivo de este tp será clasificar mails en spam y no spam, para ellos, comenzaremos el trabajo seleccionando atributos adecuados que concideramos dividen bien el problema.

\subsection{Selección De atributos}

Para la selección de atributos observamos parte del json entregado en busca de palabras claves que pudieran distinguir entre spam y no spam (desde ahora, ham).

De este analisis se encontró que palabras tales como $viagra$ o $nigeria$ son muy frecuentes en los mensajes de spam, asi tambien aquellas que hagan referencia a negocios o a dinero, por lo que incluimos atributos que cuenten las veces que son mencionados estas palabras en el cuerpo y el encabezado del mensaje. Así tambien observamos que los mails de spam tienden a tener enlaces hacia sitios web o contenido html en el cuerpo del mensaje por lo que tambien contavilizamos la cantidad de ocurrencias de palabras tales como html o http y simbolos especiales como la barra invertida, o el hashtag.

De esta menera reunimos al rededor de $100$ atributos que utilizaremos a continuación para la experimentación.

\subsection{Experimentación}

Los clasificadores que utilizaremos para experimentar en este trabajo serán: 
\begin{itemize}
\item	Arboles de Deciciones
\item	Naive bayes Multinomial
\item	Vecinos mas Cercanos
\item	SVC
\item	Random Forest
\end{itemize}