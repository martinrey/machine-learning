No pudimos terminar el trabajo práctico como quisimos por malos manejos con los tiempos que nos dieron y porque no pudimos combinarlos de la mejor forma entre los integrantes del grupo (a causa de parciales, trabajo, entre otras cosas). Sin embargo, dejamos escrito aquí qué es lo que nos parece que falta y hubiera aportado mucho a este trabajo:

Elegimos usar Random Forest ya que en el siguiente paper [1] obtienen buenos valores de Accuracy, Precision y Recall. Tambien encontramos varias comparaciones entre naive bayes y random forest. En el siguiente paper [2] hay una comparacion entre que tipo de naive bayes mejoraba la clasificacion de spam. Estos eran Bernoulli multi variada NB y Multinomial NB. Concluyen que el que mejor clasificaba era Multinomial NB pero nosotros no logramos adaptar los hiperparametros para obtener resultados que se asemejen debido a la falta de tiempo. Llegamos a separar a partir de los datos que nos brindaron, un subconjunto que sea una caja negra para luego de configurar todos los hiperparametros de los clasificadores, usarlos y quedarnos con el que mejor accuracy brinde. Nos hubiera gustado calcular las matrices de confusion y ROC (como asi tambien la accuracy, precicion y recall) sobre estos resultados obtenidos clasificando dicho subconjunto.


[1] http://www.irdindia.in/journal_ijacect/pdf/vol2_iss4/1.pdf
[2] http://www.aueb.gr/users/ion/docs/ceas2006_paper.pdf