En este trabajo práctico nos propondremos clasificar mails en \textit{spam} como en no \textit{spam} (conocido también como \textit{ham}).

Para esto analizaremos los datos en busca de atributos útiles y luego experimentaremos con diversos clasificadores con la intención de encontrar los que mejor clasifiquen.

Además utilizaremos técnicas de reducción de dimensionalidad con la intención de mejorar aún más los modelos de los puntos anteriores.

Para tener una buena métrica de los resultados obtenidos, apartamos una porción del set de datos que nos fue entregado para ser utilzado al final de la experimentación y así tener una visión realista de qué tan buenos son los clasificadores elegidos. Con el resto de los datos, realizamos \textit{kfolds} con $k = 10$ con la intención de minimizar en cierta medida el \textit{overfitting} de datos.