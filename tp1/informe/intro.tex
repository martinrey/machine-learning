En este trabajo practico nos propondremos clasificar mails en spam como en no spam (conocido tambien como ham).

Para esto analizaremos los datos en busca de atributos utiles y luego experimentaremos con diversos clasificadores con la intención de encontrar los que mejor clasifiquen.

Ademas utilizaremos tecnicas de reduccion de dimencionalidad con la intención de mejorar aun mas los modelos de los puntos anteriores.

Para tener una buena metrica de los resultados obtenidos, apartamos parte del set de datos que nos fue entregado para ser utilzado al final de la experimentación y asi tener una visión realista de que tan buenos son los clasificadores elegidos. Con el resto de los datos, realizamos kfolds con k igual a diez con la intención de minimizar en cierta medida el overfiting de datos.