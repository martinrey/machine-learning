En este trabajo practico buscamos modelar el juego de 4 en linea y jugadores que utilicen técnicas de q-learning. Además exploraremos distintas políticas de exploración para utilizar en un jugador y veremos que impacto tiene cada una de ellas en los resultados obtenidos. 

Para finalizar, realizaremos distintos experimentos con el objetivo de observar de manera empírica que sucede al variar parámetros tales como el coeficiente de aprendizaje, la inicialización de la matriz Q o entrenándolo contra otros tipos de jugadores.

Algo interesante para notar aquí es que, en este juego se contará con la participación de dos agentes que compiten entre ellos, por lo que deberá prestarse particular atención en como se modelará la etapa de recompensa del algoritmo.