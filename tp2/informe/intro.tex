En este trabajo practico buscamos modelar el juego de 4 en linea y jugadores que utilicen tecnicas de q-learning. Ademas exploraremos distintas politicas de exploración para utilizar en un jugador y veremos que impacto tiene cada una de ellas en los resultados obtenidos. 

Para finalizar, realizaremos distintos experimentos con el objetivo de observar de manera empirica que sucede al variar parametros tales como el coefciente de aprendisaje, la inicialización de la matriz Q o entrenandolo contra otros tipos de jugadores.

Algo interesante para notar aqui es que, en este juego se contará con la participación de dos ajentes que compiten entre ellos, por lo que deberá prestarse particular atención en como se modelará la etapa de recompensa del algoritmo.