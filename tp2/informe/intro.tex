En este trabajo practico buscamos modelar el juego de 4 en linea, modelar jugadores que puedan jugar sobre ese modelo y por ultimo, implementar una clase jugador que utilice tecnicas de q-learning para observar su comportamiento al variar parametros o entrenandolo bajo ciertas circunstancias particulares que concideremos interesantes.

Algo interesante para notar aqui es que, en esta experiencia \buscar_mejor_palabra se cuenta con la participación de dos ajentes que compiten entre ellos, por lo que deberá prestarse particular atención en donde se modelará la etapa de recompensa del algoritmo. \completar

Otra idea que intentaremos explorar aqui es que politica de exploración utilizar en nuestro algoritmo. En particular en estre trabajo veremos que sucede al utilizar las estretegias:
\begin{itemize}
\item estrategia greedy: toma un camino random con probabilidad $\epsilon\%$ y en caso contrario utilice el mejor brazo conocido.
\item estrategia $\epsilon$-first: toma un camino random en las primeras $\epsilon$ iteraciones y luego toma el mejor camino conocido.
\item estrategia softmax: basada en una formula probabilistica que desarrollaremos mas adelante.
\end{itemize}

