Como conclusión a este tp, podemos decir que el aprendizaje por refuerzos presenta posibilidades muy interesantes a la hora de entrenar a un autómata, quitando la necesidad de implementar una lógica explicita para el algoritmo o reglas fijas, esta manera de buscar una solución resulta mucho mas maleable y adaptativa. Como desventaja de esta técnica, pudimos ver de manera experimental que resulta muy sensible a los parámetros de aprendizaje y a la manera en la que aprende, debiendo elegirse con cuidado estos parámetros.

Como trabajo futuro sería interesante implementar y experimentar en profunidad con las heristicas propuestas en el apartado de Modelado De Jugadores, que presentaba ciertas ideas interesantes y que posiblemente ayudaran a mejorar la performance.